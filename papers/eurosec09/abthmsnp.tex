% abthmsnp.tex
% Application-based TCP Hijacking
% Author: Oliver Zheng / Jason Poon / Konstantin Beznosov

\documentclass{sig-alternate}

\begin{document}

% --- Metadata ---
\conferenceinfo{EuroSec}{'09 Nuremberg, Germany}
\CopyrightYear{2007}
%\crdata{0-12345-67-8/90/01}
% --- End Metadata ---

\title{
Application-Based TCP Hijacking
}
\subtitle{Case Study on Windows Live Messenger}

\numberofauthors {3}
\author {
	\alignauthor
	Oliver Zheng\\
		\email{eurosec@oliverzheng.com}
	\alignauthor
	Jason Poon\\
		\email{mr.j.poon@gmail.com}
	\alignauthor
	Konstantin Beznosov\\
		\email{beznosov@ece.ubc.ca}
}

\date{\today}

\maketitle

\begin{abstract}
We present Application-Based TCP Hijacking (ABTH), a technique that exploits flaws within the transport and application protocols to inject data into an application session without either end noticing it. Following the injection of a TCP packet, ABTH is used to re-synchronize the TCP stacks of both the server and the client. To demonstrate the effectiveness of ABTH, we present an attack on Windows Live Messenger in which we use ABTH to perform command spoofing and user impersonation. Due to its generic nature, ABTH can be mounted on a variety of application protocols. We also propose specific countermeasures to thwart and/or limit ABTH, such as strict Ethernet switching prevention and cryptographic protection of messages.
\end{abstract}

\terms{Security, Theory}

\keywords{TCP hijacking, application-based TCP hijacking, Windows Live Messenger, application protocols, packet injection}

\section{Introduction}

Since its first specification in 1974 \cite{rfc:tcp}, the Transmission Control Protocol (TCP) has grown to become the core transport protocol for a vast number of applications including HTTP, FTP, SMTP, and TELNET.
The security properties of these application protocols partially depend on the security of TCP and the underlying Internet Protocol (IP).
Even though TCP does not offer authentication, confidentiality, or integrity, application protocols that employ TCP are expected to offer a reasonable level of security.
Many network attacks have shown prominence over the past decades in regards to vulnerabilities of the TCP design \cite{harris:tcpattacks}.
While preventive mechanisms have been developed to throttle or even eliminate most of these attacks \cite{dubrawsky:layer2}, the last item on the list of TCP vulnerabilities is yet to be written.

In this paper, we present and suggest countermeasures for a new way of attacking a TCP-based communication.
Our technique extends TCP hijacking \cite{stamp:infosec} by meddling with application-layer protocols.
Traditional TCP hijacking attacks exploit vulnerabilities of the transport and network layers.
However, the majority of these attacks have been circumvented through the use of hardware switches and routers \cite{dubrawsky:layer2}, which provide countermeasures against these direct low-level attacks.
On the other hand, Application-Based TCP Hijacking (ABTH), the exploit method presented in this paper, utilizes loopholes in the logistics of application-level communication to evade policy enforcement for the transport and IP layers.
Trivial design features of application protocols become fatal vulnerabilities that can be exploited by ABTH.

To demonstrate the feasibility and effectiveness of ABTH, we present a case study of attacking communications of Microsoft Windows Live Messenger (WLM).
With instant messaging (IM) quickly becoming ubiquitous at home and at work \cite{aol:survey}, WLM represents one of the largest IM networks, with 27 million users in the US alone \cite{microsoft:advertising}.

We show that ABTH can be used to compromise the privacy and confidentiality of WLM users.
By attacking Microsoft Notification Protocol (MSNP) using ABTH, an attacker is able to spoof any command available to a WLM client and impersonate any contact known to the victim.
As a result, unauthorized messages can be delivered to various contacts.
Such an attack could result in at minimum user inconveniences or embarrassment but in extreme cases it could even lead to devastating results.

Even though the reported case study is limited to WLM, we believe that due to its generic nature, Application-Based TCP Hijacking can be mounted on a variety of application protocols.

Among several ways to circumvent ABTH, application protocols could encrypt packets and thus put off TCP security exploits.
Internet service providers (ISPs) could employ stricter security controls on the network layer.

The remainder of the paper is organized as follows.
Relevant TCP information and Existing attacks on TCP are discussed in Section 2.
The theory and general operation of ABTH is discussed in Section 3.
The applicability of ABTH on MSNP to spoof a command and impersonate a contact is demonstrated in Section 4.
The limitations of ABTH and countermeasures against it are discussed in Section 5.
The paper is concluded in Section 6.

\section{Background and Related Work}

In this section, we provide background on TCP necessary for understanding the rest of the paper. Additionally, details of existing security flaws are presented.

\subsection{Overview of TCP}

TCP is a connection-oriented transport protocol that guarantees reliable in-order delivery of network packets \cite{rfc:tcp}.
A pair of hosts initiate contact and communicate by sending packets to each other.
Each end of the connection is identified by an IP address and a TCP port, both of which are determined prior to establishment of and maintained throughout the connection.

To provide reliable in-order delivery, each packet is tagged with a sequence number, an acknowledgement number, and a receive window, hereon after referred to as seqnum, acknum, and rcvwnd, respectively.
seqnum represents the n-th byte of data transferred; acknum confirms n-th byte of data received; rcvwnd corresponds to the number of bytes the host is willing to receive and capable of processing.
Packets containing data set the SYN and ACK flags; packets with no data only set the ACK flag to denote an empty acknowledgement packet.
For every data packet sent, an acknowledgement packet has to be received to affirm packet delivery.
A data packet with the same acknum may be received in lieu of an empty acknowledgement packet, in which case this packet needs confirmation as well.

The seqnum of one host must match acknum of the other and vice versa for the two hosts to be in a synchronized state, in which data packets can be received and processed as valid packets.
As an example to illustrate seqnum and acknum, a client connects to a server through TCP. After establishing a connection, assume the client is seqnum 50 and the server is seqnum 100.
The next packet the client sends must entail seqnum 50 and acknum 100.
If the client sends a data packet of 10 bytes, the client seqnum increases to 60 and the server must send an acknowledgement packet with seqnum 100 and acknum 60.

In the event that either host receives a packet containing an unexpected seqnum, two cases dictate the outcome.
In the first, the received seqnum is within the range of the expected seqnum and the rcvwnd; that is, the packet received probably arrived before another packet with the expected seqnum.
In such a case, the data is buffered and no acknowledgement packet is sent. (An acknowledgement packet would confirm the reception of the missing packet.)
Otherwise, the packet is dropped and an acknowledgement packet is sent with an acknum of the expected seqnum.
As shown later, this mechanism incites the undesirable effect (for the attacker) of a TCP ack storm \cite{anderson:ackstorm}.

\subsection{Attacks on TCP}

Many attacks on TCP exploit vulnerabilities of the seqnum and acknum synchronization mechanism.
We discuss some attacks pertinent to ABTH first in order demonstrate the usefulness of ABTH.
All of these attacks, including ABTH, assume the threat model that the attacker is able to listen to network traffic of a TCP session and inject spoofed packets into the network.
None of these attacks, including ABTH, aims to disable either host, for example, by launching a denial-of-service attack; rather, the aim of all these attacks is to circumvent detection and compromise systems without knowledge of the victims.
Thus, the attacker cannot delete or reroute network traffic.

\subsubsection{TCP Hijacking}

By eavesdropping on a TCP session, an attacker can observe seqnums and acknums of both hosts and can inject a spoofed TCP packet \cite{harris:tcpattacks}.
The spoofed packet would contain the numbers expected by the recipient and the source address of the the other host.
Although the packet was sent by the attacker, the recipient has no way of authenticating the source of the packet and thus would accept it as a valid packet.
However, following this spoof, the connection is effectively broken as the expected seqnums and acknums of the two hosts are out of sync.
Data packets sent to either host would be regarded as invalid due to the mismatch of numbers, and no acknowledgement packets are sent back.
Thus, the connection is quickly reset by both hosts.
Both host would notice a disruption in network service and may suspect an attack.

\subsubsection{Man-in-the-Middle}

Following a TCP hijacking attack, the attacker can act as the man-in-the-middle, by relaying all messages from one host to the other \cite{joncheray:simpleattack, gregg:stackhack}.
Each host would receive acknowledgement for every sent packet, since the attacker is injecting a spoofed packet for each packet sent by translating seqnums and acknums on the fly.
However, a problem arises when each host receives the original packets sent by each other.
As previously mentioned, a packet is considered valid if its seqnum falls within the range of the expected seqnum and the rcvwnd.
Spoofed packets cause the spoofed host to lag behind on its seqnum, and thus packets sent from the spoofed host are not in the acceptable range.
Then for each packet the spoofed host sends, the other host drops it and sends back an acknowledgement packet with the expected seqnum and acknum trying to correct this desynchronization.
The spoofed host would try to do the same by sending an acknowledgement packet, which incites the other host to send another.
This repeated cycle of sending packets creates an ack storm, in which both hosts continuously send empty packets.

While the ack storm stops as soon as one packet is dropped due to the unreliability of the underlying physical network, it creates a massive load of network traffic \cite{joncheray:simpleattack}.
Intrusion detection systems can characterize this load with statistical analysis of empty packets on the network and notify users of a possible attack.

\subsubsection{ARP Poisoning}

One way of avoiding TCP ack storms is to stop hosts from receiving legitimate packets.
While the attacker cannot delete or reroute packets, the attacker can utilize Address Resolution Protocol (ARP) poisoning to mislead hosts into ignoring legitimate packets.
ARP provides translation of IP address to MAC address.
If ARP poisoned with the wrong MAC address, a host will send TCP/IP packets to the wrong MAC address and ignore incoming packets from the real MAC address.
The attacker needs only to ARP poison one host to circumvent ack storms, as the compromised host will not respond to the other host.

However, ARP traffic does not travel beyond the local area network (LAN) and imposes the restriction that the attacker be within the same LAN as at least one host.
More importantly, ARP poisoning is a low network level attack that can be easily prevented with hardware enforcement \cite{spangler:sniffing}.
Most hardware switches disable ARP poisoning.

\subsubsection{Restoring Synchronization}

The attacker can try to resynchronize both hosts by inciting the lagging host to send packets manually \cite{lam:resync}.
As an example on the TELNET protocol, the attacker may send a notice to the user to type in a few characters.
However, slightly more experienced users will easily detect such a peculiarity as an attack.

As discussed, existing attacks on TCP do not provide much guarantee for an attack to be executed unnoticed.

\section{Application-Based TCP Hijacking}

ABTH improves upon conventional TCP session hijacking by allowing an attacker to automatically resynchronize TCP seqnums and acknums of the two hosts of the connection.
It is a specific way of exercising TCP hijacking based on the application protocol without utilizing ARP poisoning or any other techniques below the transport layer.
By exploiting vulnerabilities in the combination of TCP and application protocol, ABTH allows an attacker to send a spoofed packet and quickly restore TCP synchronization without victims detecting such event.
The technique assumes the same threat model as TCP hijacking described in the background section.

Many application protocols intended for two-way communication, such as TELNET and FTP, maintain a persistent TCP connection.
To ensure the connection is open, a host periodically sends application ping commands to the other host.
Application-level commands, such as ping, feature two characteristics of particular importance: they prompt automatic response from the other host and do not modify application states.
These two characteristics give way for the attacker bring a lagging seqnnm of a host up to a desired seqnum. 

ABTH exploits these seemingly trivial commands at the TCP level to perform TCP hijacking and resynchronization.
As a ping to a host can provoke a response and thus inflate its seqnum, assuming ping responses exceed the data length of pings, the injection of certain multiples of pings directed at each host would counterbalance the difference created by an injected packet.
As such, an attacker can spoof a packet containing a malicious command and cover up the spoof by sending innocuous commands.

The order in which these packets are injected should follow a specific rule in order to avoid an ack storm.
For a packet sent by each host, the seqnum should fall within the acceptable range of the expected seqnum and rcvwnd, and the acknum should not be larger than the seqnum of the recipient host.
For the attacker, this means alternate sending packets between the two hosts for them to catch up to each other.
If followed roughly, only a few ack packets would be sent and no ack storm would occur.

Once this is achieved, the connection is repaired and new legitimate and illegitimate application commands can be received and processed successfully.
In comparison to typical TCP hijacking methods which usually lead victims suspicious of an attack, ABTH exploits the combination of TCP and application protocol and executes an attack that not only follows the specification of the TCP and application protocol but maintains a statistically low profile of network footprint.
This allows the attacker to carry out an attack, without exploits prohibited and impractical in well-controlled networks, and evade existing intrusion detection systems.
The significance of ABTH provides the attacker an exit strategy in which the attack may be executed unnoticed, if carefully calculated.

To demonstrate the feasibility and effectiveness of ABTH, we next describe our experience with employing our attack to compromise Windows Live Messenger.

\section{Case Study: Attacking Windows Live Messenger with ABTH} First released in 1999, Windows Live Messenger (WLM) has since grown to become one of the world�s most popular instant messaging (IM) services.  Current version of WLM utilizes Microsoft Notification Protocol (MSNP) to communicate to servers within the .NET Messenger Service \cite{piccard:imsecurity}.

\subsection{.NET Messenger Service}

The .Net Messenger Service consists of a centralized cluster of servers where they provide WLM clients with �Presence and Rendezvous services required for user-to-user communication� \cite{fout:insidewlm}.
Two types of servers within the cluster include the notification servers (NS) and mixers \cite{torre:wlm}.

When a user logs into their WLM account, a persistent TCP connection to the NS is opened.
The connection between the WLM client and the NS must always remain active.
A lost connection with the NS results in the client being logged out and disconnected from the messaging service.
Through this connection, the user�s contact list, presence data (e.g., user status, email notification), and access management information (e.g., forward list, reverse list) are synchronized between the NS and the WLM client.
Another function of the NS is to setup connections to the mixer server.
Mixer servers handle all IM sessions between clients. 

% Insert .NET Messenger Service Infrastructure diagram Here

\subsection{Microsoft Notification Protocol}

MSNP is a text-based application protocol employed for communications between WLM clients and servers in the .NET messenger service.
Although the protocol was first intended to be an open standard \cite{fout:insidewlm}, it has since become proprietary and has undergone numerous revisions.
Due to the unencrypted nature of the protocol, attempts to reverse-engineer MSNP have been proven successful \cite{hypothetic:msnp, msnfanatic:msnp}.

The protocol consists of a series of UTF-8 and URL/XML encoded commands. Each MSNP command is represented by three capital letters (e.g., command PNG represents ping).

% Insert table of commands here

MSNP contains a design feature of allowing asynchronous bi-directional pings; both the client and the NS have the ability to initiate a ping.
Pings from the client to the NS are achieved through the PNG command; the NS, on the other hand, uses the CHL command to ping the client.
In order to avoid disconnection, the receiver of the ping is required to respond; disconnection typically occurs within a few seconds if a response to the ping was not received.
Note that the ping-response initiated by the NS entails three transactions (CHL-QRYQRY).
Table 1 specifies the structure of those MSNP commands that are relevant to our ABTH-based attack on the WLM.

In the official MSNP IM application, the client pings the NS approximately every 45 seconds, and approximately every minute in alternative MSNP IM applications such as Pidgin \cite{pidgin:url}.
Within this timeframe, ABTH can exploit these two sets of ping-response commands to eliminate the discrepancy between the sequence and ack numbers of the client and the NS TCP stacks created by the injection of an MSNP command.
Although some strings within the MSNP commands contain pseudo-random numbers or hash values (e.g., challenge string, MD5 fingerprint), we found that the exact these values do not have to be correct for the purpose of mounting an ABTH attack.

In addition, MSNP commands are terminated by carriage return ($\backslash$r$\backslash$n or 0x0d0a), which has a size of 2 bytes.
The specification�s flexibility with white-space padding greatly simplifies the calculations and allows reducing the number of packets necessary to re-synchronize the underlying TCP connection.
By applying the ABTH technique, an attacker has the capability of injecting a spoofed MSNP command without disconnecting the underlying TCP session.

In what follows, we describe two examples of spoofing MSNP commands without causing disconnection of the MSNP connection: display name spoofing and identity spoofing.
To demonstrate the feasibility of ABTH, we implemented both attacks using real WLM deployment and official Microsoft WLM clients.
In the second attack, we implemented our own rogue WLM mixer server, which the official Microsoft WLM clients were connected to.

\subsection{Display Name Spoofing}

In this section, we describe how we were able to modify a victim�s display name to an arbitrary string.
For this attack, the MSNP command PRP was employed; its format is shown in Table 1.

We spoofed a PRP command to the NS.
Once the packet was received by the NS, it responded to the client by sending back the exact same command but without the additional white space that was added in the original PRP following the carriage return.
The NS then updated the user�s information resulting in presence updates pushed to all of the user's contacts.
While all of the user's contacts were notified of the change, the victim user was completely oblivious to the situation and did not notice the change in their display name locally.

Following the injection of the packet, the TCP stacks of the WLM client and the NS were resynchronized through ABTH.
For demonstration purposes, the sequence numbers of the client and the NS are assumed to be 100 and 500 prior to the spoofed packet.
During normal operation, NS ack number should match client sequence number and client ack number should match NS sequence number, as seen in the first row of Table 2.

% Insert packet exchange for display name spoofing table here

While we could not control the sequence number of the NS or the client, we could alter the ack numbers of both sides by spoofing a whitespace-padded packet.
In this case, the number of spoofed packets to send to each side of the connection was pre-calculated.
The fact that MSNP commands allow white-spacepadding greatly simplified our calculations.
Table 3 shows the amount of padding that used for this example.
Each command was sent or received by the attacker in the order listed in Table 2.

As the last row of Table 2 shows, the client sequence number was equal to the NS ack number, and the NS sequence number was equal to the client ack number.
ABTH had completed its series of application specific commands and the TCP connection had returned to a synchronized state and was ready to transmit further legitimate and illegitimate packets.

% Insert padded commands here

\subsection{Identity Spoofing}

By exploiting the same vulnerability, ABTH can be used to spoof the identities of WLM participants.
As illustrated in Figure 2, when a user wishes to initiate a conversation with another contact, the user first requests an IM session through the NS.
The NS returns a session authentication ID and network destination IP address and port that points to a mixer.
The client then establishes a persistent TCP connection to the designated mixer.
Once the client�s connection to the mixer has been established, the client proceeds to invite other contacts to join the conversation.
Each invited contact receives an invite through their NS connection to connect to the specified mixer.
Once all parties are connected to the mixer, it begins to relay all instant messages between the WLM clients.

% Insert IM Initiation sequencing diagram here

In order to spoof an instant messaging session, we sent a spoofed invite message to the victim as the NS, which correlated to step 4 of Figure 2, instructing the victim to connect to a spoofed mixer.
As shown in Table 1, the invite message contained the address of the spoofed mixer server and the identity of the contact that we wished to impersonate, in this case Bob.
When the victim, Alice, connected to our own mixer server, the authentication string and the type of authentication (CKI) specified in the invite message was ignored.

Upon receiving the invite message, the victim established a connection to the address specified in the invite message, which in this case was a rogue mixer server.
The TCP connection with the NS was now out of order and Alice would soon be disconnected and unable to communicate with the attacker unless the connection was resynchronized.
Thus, following the injection of the RNG command, we employed ABTH to mask the effects of the injected packets.
Our calculations indicated a minimum spoofing of approximately 30 packets, as shown in Table 4 and Table 5.

% Insert padded commands table here

% Insert packet exchange of identity spoofing table here

In cases such as this, it is crucial that ABTH is completed prior to the next client ping, which occurs in roughly 45 seconds.
Otherwise, the client would repeatedly ping the NS and very quickly timeout before more packets could be sent to fix the imbalance in the sequence numbers.
Depending on the resources of the attacker, this attack can be reasonably accomplished within this timeframe, as our experiments demonstrated.

Once connected to the victim, our rogue mixer server then sent the identities of the participants already in the session through the IRO command, which only included Bob, as shown in Table 1.
At this point, we had established a separate and legitimate TCP connection with the victim, who believed the other end of the connection was an authenticated mixer.
No additional attack packets were required to send messages to Alice on behalf of Bob, beyond sending correctly formatted MSNP commands.

The victim�s WLM client recognized the participant�s email and display name as a contact on their contact list and assumed that an authentic instant messaging session had been established with Bob (msnp\_bob@hotmail.com).
We could now send any messages to the victim through this spoofed connection and there was not much Alice could do to reveal Bob�s real identity.

\section{Discussion}

Our experiments demonstrate that MSNP could be successfully attacked by employing ABTH.
The technique, however, has several caveats.

\subsection{Limitations}

ABTH requires the length of responses be greater than the commands issued to provoke them.
If this was not the case, resynchronization would only widen the gap of mismatched seqnums.
Similarly, application commands with similar behaviour to NOP (no operation) that do not require responses would not work either.

Our implementation of ABTH on the MSNP protocol calculates and generates the required commands to send prior to the attack and are therefore static.
If either host sends packets during the restoration phase, this would result in ABTH to fail in resynchronizing their TCP stacks.
As a result, no network traffic can occur while ABTH is in the process of sending its own packets.
Complex algorithms capable of adjusting to live traffic dynamically may improve this performance.

\subsection{Countermeasures}

A straight-forward countermeasure against ABTH is to encrypt all application traffic with SSL/TLS to authenticate messages.
However, large-scale applications such as MSNP would require substantial server resources in encrypting all network traffic.

Some applications may connect peers to peers, as MSNP does implicitly by allowing a communication channel between two clients.
In such cases, clients may encrypt traffic and authenticate each other through the channel provided by the application.
As a case in point, SimpLite-MSN \cite{secway:url}, according to its marketing description, sets up RSA keys and authenticates all instant messaging traffic for MSNP, providing protection against spoofing.

Intrusion detection systems (IDS) may possibily detect ABTH given a well characterized behaviour of attacks.
Since ABTH utilizes minimal resources and is limited to the specification of TCP and the application, the behaviour would need to be described precisely, which may or may not be feasible as it may step within boundaries of legitimate application traffic.

Even if packets are sent in plain text, a simple regulatory mechanism of the application protocol could present some integrity with each message, such as tagging application messages with sequence numbers.
It would be difficult to inject a packet and expect further messages to be accepted since the application also keeps track of the list of messages.
This straightforward yet non-trivial detail also alleviates the application protocol from its dependence on the network layer and the assumption that packets on the network layer are authentic.

For network service providers, the most effective method to disable ABTH and message spoofing all together would be to restrict IP packets containing designated source IP addresses for each port of a switched network \cite{templeton:spoof}.
However, this method would not ineffective in a wireless medium, such as WiFi.
Of course, WiFi traffic could be encrypted to circumvent ABTH.

\subsection{Feasibility}

Our particular implementation of ABTH on MSNP completes within roughly 2 seconds.
It can be launched on the trigger of capturing an MSNP ping packet.
In our findings, the probability of network traffic occurring within 2 seconds of a ping is less than 5\%, as logged by several days of network traffic.
As such, ABTH would prevail over 95\% of the time, a large concern for potential threat.

In environments that provide a common physical communication medium, ABTH may present a threat.
Examples include hubbed networks, coffee shops that offer WiFi service, and increasingly popular virtualized servers.
Application protocols vulnerable to ABTH, such as MSNP, were not designed to offer secure communication channels.
It is imperative that, as protocols become more widely adopted, they are improved upon and revised to accommodate their broadening uses.

\section{Conclusion}

In this paper, we introduced Application-Based TCP Hijacking (ABTH).
By combining the innocuous features of the transport and application layers and the lack of cryptographic protection at either layer, ABTH offers an elegant way of exploiting certain application protocols.
By sending commands to both hosts to provoke responses, an attacker can create a gap of a certain size in TCP sequence numbers to inject a malicious command.
We presented a proof of ABTH concept by successfully experimenting with two sample attacks on Windows Live Messenger, Microsoft's popular instant messaging application, which runs over Microsoft Notification Protocol (MSNP).

While some protocols, such as MSNP, were designed to suit flexible environments, as the environment evolved, these features have become subtle vulnerabilities.
Although hardware Ethernet equipment has been updated to obstruct lower level attacks (e.g., ARP poisoning), the realm of attacks is reaching beyond the protection capabilities of network devices.

\bibliographystyle{abbrv}
\bibliography{abthmsnp}

\end{document}
